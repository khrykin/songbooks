% XeLaTeX can use any Mac OS X font. See the setromanfont command below.
% Input to XeLaTeX is full Unicode, so Unicode characters can be typed directly into the source.

% The next lines tell TeXShop to typeset with xelatex, and to open and save the source with Unicode encoding.

%!TEX TS-program = xelatex
%!TEX encoding = UTF-8 Unicode

\documentclass[14pt]{article}
%\usepackage[chordbk]{songbook} 
\usepackage{gchords}  
\usepackage{geometry}
\usepackage{rotating}
\usepackage{multicol}
\usepackage{enumitem,xcolor}
\usepackage{hyperref}
\hypersetup{
    colorlinks,
    citecolor=black,
    filecolor=black,
    linkcolor=black,
    urlcolor=black
}

\geometry{a4paper}                   % ... or a4paper or a5paper or ... 
%\geometry{landscape}                % Activate for for rotated page geometry
%\usepackage[parfill]{parskip}    % Activate to begin paragraphs with an empty line rather than an indent
\usepackage{graphicx}
\usepackage{amssymb}

% Will Robertson's fontspec.sty can be used to simplify font choices.
% To experiment, open /Applications/Font Book to examine the fonts provided on Mac OS X,
% and change "Hoefler Text" to any of these choices.

\usepackage{fontspec,xltxtra,xunicode}
\newcommand{\up}[1]{\textsuperscript{#1}}

\renewcommand\thesection{}

\defaultfontfeatures{Mapping=tex-text}
\setromanfont[Mapping=tex-text]{Hoefler Text}
\setsansfont[Scale=MatchLowercase,Mapping=tex-text]{Gill Sans}
\setmonofont[Scale=MatchLowercase]{Andale Mono}
\title{Different Class}
\author{Songbook}
\date{}                                           % Activate to display a given date or no date
% Chords 

% C
\newcommand{\CMaj}{\chord{t}{x, p3, p2, o, p1, o}{C}}
\newcommand{\CMajSeven}{\chord{t}{x, p3, p2, o, o, o}{C\up{maj7}}}
\newcommand{\CSeven}{\chord{t}{x, p3, p2, p3, p1, o}{C\up7}}

% D

\newcommand{\DMaj}{\chord{t}{x, x, o, p2, p3, p2}{D}}
\newcommand{\DSeven}{\chord{t}{x, x, o, p2, p1, p2}{D\up7}}

% E

\newcommand{\EMaj}{\chord{t}{o, p2, p2, p1, o, o}{E}}



% F#

\newcommand{\Fshm}{\chord{t}{p2, p4, p4, p2, p2, p2}{F\#m}}


%G
\newcommand{\GMaj}{\chord{t}{p3, p2, o,o, o, p3}{G}}
\newcommand{\GMajSeven}{\chord{t}{p3, p2, o,o, o, p2}{G\up{maj7}}}
\newcommand{\GSeven}{\chord{t}{p3, p2, o,o, o, p1}{G\up 7}}

%A

\newcommand{\AMaj}{\chord{t}{x, o, p2, p2, p2, o}{A}}

\newcounter{SongVerseCnt}

\newenvironment{myenumerate}{%
  \edef\backupindent{\the\parindent}
  \begin{enumerate}[label=\color{gray}\theenumi]
  \setlength{\parindent}{\backupindent}
}{\end{enumerate}}


\newenvironment{song}[2]
{
	\setcounter{SongVerseCnt}{0}
	%\keyIndex{#1}{\theSBSongCnt}

		\begin{center}
			\section{#1}
			#2
		\end{center}
	\begin{multicols}{2}
\normalsize
\begin{myenumerate}
}{
\end{myenumerate}
\end{multicols}
\newpage
}


\definecolor{blue}{rgb}{0.2, 0.5, 0.5}

\newenvironment{SongVerse}
{
	%\stepcounter{SongVerseCnt}
	\item
	\setlength{\parindent}{0cm}
}{
\newcommand{\SBChordRaise}{2.25ex}
}
\newcommand{\ch}[2]{
	\makebox[0pt][l]{\raisebox{2.5ex}{\color{blue}\sffamily \,\,#1}}
	\makebox{#2}
}


% \def\numfrets{4}

\begin{document}

\maketitle

\newpage

\tableofcontents 
\newpage

\begin{song}{Mis-Shapes}{

    \git{Verse}

	\mbox{ \AMaj \EMaj \Fshm \DMaj \DSeven}

    \git{Chorus}

	\mbox{ \GMaj \GMajSeven \GSeven \CMaj \CMajSeven \CSeven }
	

	\mbox{ \Em \EmSix \EmaddC }

}

	 \begin{SongVerse}
		\ch{A}{Mis-shapes}, mistakes, misfits. 

		\ch{E}{Raised} on a diet of broken biscuits, oh \ch{F\#m}{}

		We don't look the same as you \ch{D}{}
		
		We don't do the things you do,
		
		But \ch{D7}{we} live around here too, oh really. 
	 \end{SongVerse}

	 \begin{SongVerse}
		\ch{A}{Mis-shapes}, mistakes, misfits,

		We'd \ch{E}{like} to go to town but we can't risk it, oh \ch{F\#m}{} 

		'Cos they just want to keep us out. \ch{D}{} 
		
		You could end up with a smack in the mouth

		\ch{D7}{Just} for standing out, now really.
	 \end{SongVerse}

	 \begin{SongVerse}

		\ch{A}Brothers, sisters, can't you \ch{E}see? 

		The future's owned by you and \ch{F\#m}me. 

		There won't be fighting in the \ch{D}street. 

		They think they've got us beat, 

		But \ch{D\up7}revenge is going to be so sweet. 
	\end{SongVerse}

	\begin{SongVerse}
		
		\ch{G} \quad We're making a \ch{G\up{maj7}}move, 

		we're making it \ch{G\up 7}now. 

		We're coming out of the sidelines. 

		\ch{C}\quad Just put your \ch{C\up{maj7}}hands up -- it's a \ch{C\up7}raid \ch{C}... \ch{C\up7}yeah.

		We want your \ch{Em}homes,

		we want your \ch{Em\up{addC}}lives,

		we want the \ch{Em6}things you won't \ch{Em\up{addC}}allow us. 

		We won't use \ch{Em}guns, 

		we won't use \ch{Em\up{addC}}bombs

		We'll use the \ch{Em6}one thing we've got \ch{Em\up{addC}}more of --

		that's our \ch{Em}minds. \ch{Em\up{addC}} \hspace{20pt} \ch{Em6} \qquad \ch{Em}

	\end{SongVerse}

	\begin{SongVerse}

		\ch{A} Check your lucky numbers.

		\ch{E}That much money could drag you under, oh. \ch{F\#m}

		What's the point of being rich \ch{D}

		if you can't think what to do 
		
		with it 

		'cos \ch{D\up7}you're so bleeding thick?

	\end{SongVerse}

	\begin{SongVerse}

		
		\ch{A} \quad Oh, we weren't supposed to \ch{E}be -- 

		we learnt too much at school now 

		\ch{F\#m} we can't help but see 

		that the \ch{D}future that you've got mapped out 

		is \ch{D\up7} nothing much to shout about. 

	\end{SongVerse}

	\begin{SongVerse}

		
		\ch{G} \quad We're making a \ch{G\up{maj7}}move, 

		we're making it \ch{G\up 7}now. 

		We're coming out of the sidelines. 

		\ch{C}\quad Just put your \ch{C\up{maj7}}hands up -- it's a \ch{C\up7}raid \ch{C}... \ch{C\up7}yeah.

		We want your \ch{Em}homes,

		we want your \ch{Em\up{addC}}lives,

		we want the \ch{Em6}things you won't \ch{Em\up{addC}}allow us. 

		We won't use \ch{Em}guns, 

		we won't use \ch{Em\up{addC}}bombs

		We'll use the \ch{Em6}one thing we've got \ch{Em\up{addC}}more of --

		that's our \ch{Em}minds. \ch{Em\up{addC}} \hspace{20pt} \ch{Em6} \qquad \ch{Em}

	\end{SongVerse}

	\begin{SongVerse}
				\ch{A}Brothers, sisters, can't you \ch{E}see? 

		The future's owned by you and \ch{F\#m}me. 

		There won't be fighting in the \ch{D}street. 

		They think they've got us beat, 

		But \ch{D\up7}revenge is going to be so sweet. 
	\end{SongVerse}

	\begin{SongVerse}		

		
		\ch{G} \quad We're making a \ch{G\up{maj7}}move, 

		we're making it \ch{G\up 7}now. 

		We're coming out of the sidelines. 

		\ch{C}\quad Just put your \ch{C\up{maj7}}hands up -- it's a \ch{C\up7}raid \ch{C}... \ch{C\up7}yeah.

		We want your \ch{Em}homes,

		we want your \ch{Em\up{addC}}lives,

		we want the \ch{Em6}things you won't \ch{Em\up{addC}}allow us. 

		We won't use \ch{Em}guns, 

		we won't use \ch{Em\up{addC}}bombs

		We'll use the \ch{Em6}one thing we've got \ch{Em\up{addC}}more of --

		that's our \ch{Em}minds. \ch{Em\up{addC}} \hspace{20pt} \ch{Em6} \qquad \ch{Em}

	 \end{SongVerse}

\end{song}
\begin{song}{Pencil Skirt}{
	
	\chordset[Verse]{ \CMaj \AMaj \DMaj \GMaj }
	\chordset[Chorus]{ \EmShAm \CMajShE \DMajShA \GMajShE}
	\vspace{1em}
	\tabset[Riff 1]{\includegraphics{tab/pencilskirt1}}%
	\tabset[Riff 2]{\includegraphics{tab/pencilskirt2}}
}


\begin{songverse}


\ch{Riff 1}\hspace{30pt}\ch{C}When you raise your pencil \ch{A}skirt like a 

\ch{D}veil before my \ch{G}eyes \ch{Riff 2}

 \ch{C}Like the look upon his \ch{A}face as he's

 
 \ch{D}zipping up his \ch{G}flies, \ch{Riff 2}oh


 \ch{C}But I \ch{A}know that


 \ch{D}you're engaged to \ch{G}him  \ch{Riff 2}, oh


\ch{C}\hspace{10pt}But I  \ch{A}know you \ch{D}want something 


to \ch{G}play with, baby

\end{songverse}

\begin{songchorus}

 \ch{Em}I'll be around when he's not in 
 
 town, oh \ch{C}


 Yeah I'll show you how you're 
 
 doing it wrong, oh \ch{Em}


 I really love it when you tell me 
 
 to stop, oh oh \ch{D}
                
 
 Oh it's turning me \ch{G}on

\end{songchorus}

\begin{songverse}

\ch{Riff 2}\hspace{30pt}\ch{C} Now you can tell some \ch{A}lies about the\ch{D}good times that you've \ch{G}had \ch{Riff 2}


\ch{C}But I've kissed your mother \ch{A}twice and


I'm work\ch{D}ing on your \ch{G}dad, oh baby


\end{songverse}

\begin{songchorus}

 \ch{Em}I'll be around when he's not in 
 
 town, oh \ch{C}


 Yeah I'll show you how you're 
 
 doing it wrong, oh \ch{Em}


 I really love it when you tell me 
 
 to stop, oh oh \ch{D}
                
 
 Oh it's turning me \ch{G}on

\end{songchorus}


\begin{songverse}
\ch{Riff 2}\hspace{30pt} \ch{C}If you look under the \ch{A}bed then I can \ch{D}see my house 

from \ch{G}here \ch{Riff 2}

 \ch{C}So just lie against  the \ch{A}wall and watch my  \ch{D}conscience disap\ch{G}pear, now baby oh


 \ch{Em}Yeah, I'll be around when he's not in town, oh
 
 
 \ch{C}Oh yeah, I'll show you how you're doing it wrong, oh
 
 
 \ch{Em}I really love it when you tell me to stop, oh oh
 
                  
 \ch{D}Oh it's turning me \ch{G}on, on, on, yeah

\end{songverse}


 \begin{songverse}[Bridge]

 \ch{Em}\hspace{10pt}I only come here cause I know it makes you sad
 
 
 
 \ch{C}\hspace{10pt}I only do it cause I know you know it's bad.
 
 
 
 Oh don't you \ch{Em}know it's ugly and it shouldn't be like that.
 
                          
 \ch{D}\hspace{10pt}Oh but oh it's turning me \ch{G}on, on, on, on, on, on, on.


\ch{C}  \hspace{15pt}  \ch{A} \hspace{15pt}   \ch{D}  \hspace{15pt}   \ch{G}
 \end{songverse}

\end{song}

\end{document}